\documentclass{ximera}
\input{../preamble.tex}
\author{Steven Gubkin}
\license{Creative Commons 3.0 By-NC}
\begin{document}

\begin{exercise}

\outcome{Use the product and quotient rule to calculate derivatives from a table of values.}
\outcome{Use the product rule to calculate derivatives.}
\outcome{Use the quotient rule to calculate derivatives.}
\outcome{Combine derivative rules to take derivatives of more complicated functions.}

\tag{derivative}

Let $P(x) = A(x)B(x)$

If you know that $A(0) = 0$, $A'(0) = 0$, $B(0) > 0$, and $B'(0) > 0$, what can you say about the sign of $P'(0)$?

\begin{multipleChoice}
\choice{$P'(0)>0$}
\choice{$P'(0)<0$}
\choice[correct]{$P'(0) = 0$}
\choice{We cannot determine the sign of $P'(0)$}
\end{multipleChoice}

Try to make sense of this by thinking about small increases/decreases in the quantities $A$ and $B$, in addition to symbolic calculation.

\end{exercise}

\end{document}
