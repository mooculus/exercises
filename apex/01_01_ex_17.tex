\documentclass{ximera}

\input{preamble.tex}

\author{Gregory Hartman \and Matthew Carr}
\license{Creative Commons 3.0 By-NC}
\acknowledgement{https://github.com/APEXCalculus}

\begin{document}
\begin{exercise}

\outcome{Estimate limits using nearby values.}
\outcome{Estimating limits numerically and possible errors to this method.}

\tag{limit}
\tag{derivative}

% The r@{.}l aligns everything at a common decimal point by constructing two columns and 
% aligning them together at that decimal point to create the effect that only a single column is
% there. It's a quick fix to a very minor blemish in the source.

% We need the multicolumn{2}{c} to align the 'h' over the center of the two columns we have     
% made, since we cannot just change the number of columns midway through our construction. 
% By indicating that there are 2 columns we are merging, TeX understands our construction.

% I have to  put extra columns into this one, else the RHS doesn't line up quite right. Better way to 
% do this?



Let $f(x) =\sin(x)$ and $a=\pi$. Observe the table of values for $\frac{f(a+h)-f(a)}{h}$:
\begin{center}
 \begin{tabular}{r@{.}lc@{\hspace{23pt}}r@{.}l}
  \multicolumn{2}{c}{$h$} & \multicolumn{3}{c}{$\frac{f(a+h)-f(a)}{h}$}\\ \hline 
  $-0$ & $1$ & & $-0$ & $998334$  \\
  $-0$ & $01$ & & $-0$ & $999983$ \\
  $0$ & $01$ & & $-0$ & $999983$ \\
  $0$ & $1$ & & $-0$ & $998334$
 \end{tabular}
\end{center}
The limit as $h\to 0$ is approximately $\begin{prompt}\answer{-1}\end{prompt}$.

\end{exercise}
\end{document}