\documentclass{ximera}

\input{../preamble.tex}

\author{Gregory Hartman \and Matthew Carr}
\license{Creative Commons 3.0 By-NC}
\acknowledgement{https://github.com/APEXCalculus}

\begin{document}
\begin{exercise}

\outcome{Identify the functions required to transform the integral via integration by parts.}
\outcome{Evaluate integrals using integration by parts, including multiple iterations.}
\outcome{Undo the Chain Rule.}
\outcome{Calculate indefinite integrals (antiderivatives) using basic substitution.}

\tag{antiderivatives}
\tag{integral}
\tag{substitution}
\tag{integration by parts}


%% BADBAD 
%% Absolute Values here? Integration by parts yields an \ln(x), not \ln|x| in the u*v term of 
%% \int u dv == u*v-\int v du  ! ! ! I have changed it to \ln(x) on the principle that differentiating should yield the 
%% integrand. With \ln|x|, one obtains (x-2)\ln|x|, not (x-2)\ln(x)




Find the indefinite integral of $(x-2)\ln(x)$ with respect to $x$.

\[
\int (x-2)\ln(x)\d x=
\begin{prompt}
\answer{-\frac{x^2}{4}+\frac{1}{2}x^2\ln(x)+2x-2x\ln(x)+C}
\end{prompt}
\]


\end{exercise}
\end{document}
