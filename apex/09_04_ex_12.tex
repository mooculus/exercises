\documentclass{ximera}

\input{../preamble.tex}

\author{Gregory Hartman \and Matthew Carr}
\license{Creative Commons 3.0 By-NC}
\acknowledgement{https://github.com/APEXCalculus}

\begin{document}
\begin{exercise}

\outcome{Convert between polar and Cartesian coordinates.}
\outcome{Use the Cartesian to polar method to plot polar graphs.}


\tag{polar coordinates}


Given the graph of a function in rectangular (Cartesian) coordinates below, find the equation representing the graph in polar coordinates subject to the correct restriction(s) on $\theta$ and $r$.

 \begin{center}
 \begin{tikzpicture}
 \begin{axis}[ymin=-1-1/10, ymax=2+1/10,
   axis lines=center, 
   axis equal,
   xlabel=$x$, 
   ylabel=$y$,
   every axis y label/.style={at=(current axis.above origin),anchor=south}, 
   every axis x label/.style={at=(current axis.right of origin),anchor=west}, 
   grid = major]
  \addplot[penColor, samples=400, very thick, domain=-1:2] {sqrt(3)/2*x};
  \draw[densely dashed, thick] (axis cs:2,{sqrt(3)})--(axis cs:0,{sqrt(3)});
  \node [draw,fill=black,circle,inner sep=0pt,minimum size=2.5pt] at (axis cs: 2,{sqrt(3)}) {};
  \node [label=left:$\sqrt{3}$,draw,fill=black,circle,inner sep=0pt,minimum size=2.5pt] at (axis cs: 0,{sqrt(3)}) {};
\end{axis} 
\end{tikzpicture}     
\end{center} 


\[
\theta(r)=\begin{prompt}\answer{\frac{\pi}{6}}\end{prompt}
\]
subject to the restriction(s) that 
\[
\begin{prompt}\answer{-1}\end{prompt}\le r\le \begin{prompt}\answer{2}\end{prompt}
\]

      
\end{exercise}
\end{document}