\documentclass{ximera}

\input{../preamble.tex}

\author{Gregory Hartman \and Matthew Carr}
\license{Creative Commons 3.0 By-NC}
\acknowledgement{https://github.com/APEXCalculus}

\begin{document}
\begin{exercise}

\outcome{Compute derivatives of polar curves.}
\outcome{Find horizontal and vertical tangent lines of a polar curve.}
\outcome{Determine where the derivative of a polar curve is undefined.}


\tag{polar coordinates}
\tag{derivative}


Find all values of $\theta$ in $[0,2\pi]$ for which the tangent line to the polar curve $r(\theta)=\cos(2\theta)$ is horizontal. List your values of $\theta$ from least to greatest. You will have to use a calculator.
\[
\begin{prompt}\text{$\theta$ is one of }\answer{\arctan(1/\sqrt{5})}, \answer{\pi/2}, \answer{\pi-\arctan(1/\sqrt{5})}, \answer{\pi+\arctan(1/\sqrt{5})}, \answer{\frac{3\pi}{2}}, \answer{2\pi-\arctan(1/\sqrt{5})}\ldotp\end{prompt}
 \]
 
Conversely, find all values of $\theta$ in $[0,2\pi]$ for which the tangent line to the polar curve $r(\theta)=\cos(2\theta)$ is vertical. List your values of $\theta$ from least to greatest. You will have to use a calculator.
 \[
\begin{prompt}\text{$\theta$ is one of }\answer{0}, \answer{\arctan(\sqrt{5})}, \answer{\pi-\arctan(\sqrt{5})}, \answer{\pi}, \answer{\pi+\arctan(\sqrt{5})}, \answer{2\pi-\arctan(\sqrt{5})}\ldotp\end{prompt}
 \]

 
\end{exercise}
\end{document}