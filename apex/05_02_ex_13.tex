\documentclass{ximera}

\input{preamble.tex}

\author{Gregory Hartman \and Matthew Carr}
\license{Creative Commons 3.0 By-NC}
\acknowledgement{https://github.com/APEXCalculus}

\begin{document}
\begin{exercise}

\outcome{Compute definite integrals using the properties of integrals.}
\outcome{Use integral notation for both antiderivatives and definite integrals.}
\outcome{Evaluate definite integrals using the Second Fundamental Theorem of Calculus.}
\outcome{Understand the relationship between indefinite and definite integrals.}

\tag{definite integral}
\tag{integral}
\tag{antiderivative}

Compute the following definite integrals:

\begin{minipage}[t]{.5\linewidth}
\begin{enumerate}
\item		$\int_{0}^{2}5x^2\d x=\answer{\frac{40}{3}}$
\item		$\int_{0}^{2}x^2+3\d x=\answer{\frac{26}{3}}$
\end{enumerate}
\end{minipage}
\begin{minipage}[t]{.5\linewidth}
\begin{enumerate}\addtocounter{enumi}{2}
\item		$\int_{1}^{3}(x-1)^2\d x=\answer{\frac{8}{3}}$
\item		$\int_{2}^{4}(x-2)^2+5\d x=\answer{\frac{38}{3}}$
\end{enumerate}
\end{minipage}

\end{exercise}
\end{document}