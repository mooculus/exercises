\documentclass{ximera}

\input{../preamble.tex}

\author{Gregory Hartman \and Matthew Carr}
\license{Creative Commons 3.0 By-NC}
\acknowledgement{https://github.com/APEXCalculus}

\begin{document}
\begin{exercise}

\outcome{Solve basic related rates word problems.}
\outcome{Identify word problems as related rates problems.}
\outcome{Understand the process of solving related rates problems.}

\tag{related rates}

Water flows onto a flat surface at a rate of $5$cm$^3$/s forming a circular puddle $10$mm deep. How fast is the radius growing when the radius is:
\begin{enumerate}
\item		$1$cm? \[\answer{\frac{5}{2\pi}}\,cm/s\]
\item		$10$cm? \[\answer{\frac{1}{4\pi}}\,cm/s\]
\item		$100$cm? \[\answer{\frac{1}{40\pi}}\,cm/s\]

\end{enumerate}

\begin{hint}
Recall that $1$ millimeter is $\frac{1}{10}$ centimeters, so $10$mm is $1$cm. Hence, the volume of the puddle as a function of time is $V(t)=\pi r^2$ where $r$ is the radius of the puddle as a function of time. What is the derivative of $V(t)$ with respect to $t$. What about the derivative of $r^2$ with respect to $t$, given that $r$ is a function of time?
\end{hint}
\begin{hint}
The derivative as a function of time is $5=\pi(2r\frac{dr}{dt})$ since $\frac{dV}{dt}=5$cm$^3$/s.
\end{hint}
\begin{hint}
Then $\frac{dr}{dt}=\frac{5}{2r\pi}$. For any value of $r$, we can find $\frac{dr}{dt}$ by plugging in that value.
\end{hint}
\end{exercise}
\end{document}
