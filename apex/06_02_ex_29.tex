\documentclass{ximera}

\input{../preamble.tex}

\author{Gregory Hartman \and Matthew Carr}
\license{Creative Commons 3.0 By-NC}
\acknowledgement{https://github.com/APEXCalculus}

\begin{document}
\begin{exercise}

\outcome{Identify the functions required to transform the integral via integration by parts.}
\outcome{Evaluate integrals using integration by parts, including multiple iterations.}
\outcome{Undo the Chain Rule.}
\outcome{Calculate indefinite integrals (antiderivatives) using basic substitution.}

\tag{antiderivatives}
\tag{integral}
\tag{substitution}
\tag{integration by parts}


%% BADBAD 
%% Should we be picky about the absolute values in the antiderivative of \tan(x) ? I say yes ? ? ?

Find the indefinite integral of $x\sec^{2}(x)$ with respect to $x$.

\[
\int x\sec^{2}(x)\d x
\begin{prompt}
=\answer{x\tan(x)+\ln|\cos(x)|+C}
\end{prompt}
\]


\end{exercise}
\end{document}
