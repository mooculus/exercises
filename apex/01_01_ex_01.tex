\documentclass{ximera}

\input{preamble.tex}

\author{Gregory Hartman \and Matthew Carr}
\license{Creative Commons 3.0 By-NC}
\acknowledgement{https://github.com/APEXCalculus}

\begin{document}
\begin{exercise}

\outcome{Calculate limits using the limit laws.}

\tag{limit} 
\tag{quadratic} 
\tag{continuous}
  
  
  Find 
  \[
  \lim_{x\to 1} \left(x^2+3x-5\right)
  \begin{prompt}
  = \answer{-1}.
  \end{prompt}
  \]
  
   \begin{hint}
      This function is continuous everywhere. Therefore, left-hand and right-hand limits exist at every point and are equal. Use this to your advantage by applying limit laws. Namely, the limit of a sum is the sum of the limits. Hence, 
    \[
    \lim_{x\to 1} \left( x^2+3x-5 \right)  
    = \lim_{x\to 1} \left( x^2 \right) +
    \lim_{x\to 1} \left( 3x \right) +
    \lim_{x\to 1} \left( -5 \right).
    \]
    \end{hint}
    
     \begin{hint}
    Take a look at the graph of the function
    \begin{center}
     \begin{tikzpicture}
	\begin{axis}
	[ymin=-3,ymax=3, xmin=-2, xmax=2, axis lines=center,xlabel=$x$,ylabel=$y$,every axis y 
	label/.style={at=(current axis.above origin),anchor=south},every axis x label/.style={at=(current axis.right of origin),anchor=west},
	domain=-3:3,
	yticklabels={},
	ymajorgrids=true,
	grid = major
	]
	\addplot[very thick,smooth]
	{\x^2+3*\x-5};
	\end{axis}
       \end{tikzpicture}
      \end{center}
    Additional limit laws imply that
    \[
    \lim_{x\to 1} \left( x^2+3x-5 \right)  
    = \left( \lim_{x\to 1} \left( x \right) \right)^2 +
    3 \cdot \lim_{x\to 1} \left( x \right) +
    \lim_{x\to 1} \left( -5 \right).
    \]
    \end{hint}
    \begin{hint}
     Evaluating $\lim_{x\to 1} \left(x^2+3x-5\right)  
    = \left(\lim_{x\to 1}  x \right)^2 +
    3\cdot\lim_{x\to 1} x  -
    \lim_{x\to 1} 5$
    we see the answer is $-1$, since, as we know, $\lim_{x\to1}x=1$.
    \end{hint}
    
\end{exercise}
\end{document}
