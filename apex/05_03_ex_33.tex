\documentclass{ximera}

\input{../preamble.tex}

\author{Gregory Hartman \and Matthew Carr}
\license{Creative Commons 3.0 By-NC}
\acknowledgement{https://github.com/APEXCalculus}

\begin{document}
\begin{exercise}

\outcome{Add up a large number of terms quickly using sigma notation.}
\outcome{Compute left, right, and midpoint Riemann Sums with many rectangles.}
\outcome{Understand the relationship between area under a curve and sums of rectangles.}
\outcome{Approximate area under a curve.}
\outcome{Understand how the area under a curve is related to the antiderivative.}
\outcome{Use limits of Riemann sums to find the exact area under a curve.}
\outcome{Understand how Riemann sums are used to find exact area.}


\tag{series}
\tag{Riemann sum}
\tag{integral}

Answer the following, given that it is a well known fact that  $\sum_{k=1}^{m}k^3=\frac{m^2(m+1)^2}{4}$.

\begin{enumerate}
\item		Find the right Riemann Sum for $n$ equally spaced rectangles approximating $\int_{0}^{1}x^3\d x$ \[\int_{0}^{1}x^3\d x\approx\answer{\frac{(n+1)^2}{4n^2}}\]
\item		The limit as $n\to\infty$ of your answer to part (a) is: \[\answer{\frac{1}{4}}\]
\item		Does the answer to part (b) agree with the value of the integral $\int_{0}^{1}x^3\d x$? \begin{multipleChoice}
\choice[correct]{Yes}
\choice{No}
\end{multipleChoice}
\end{enumerate}
\end{exercise}
\end{document}
