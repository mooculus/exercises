\documentclass{ximera}

\input{preamble.tex}

\author{Gregory Hartman \and Matthew Carr}
\license{Creative Commons 3.0 By-NC}
\acknowledgement{https://github.com/APEXCalculus}

\begin{document}
\begin{exercise}

\outcome{Calculate limits of the form 0/0.}
\outcome{Distinguish between limit values and function values.}

\tag{limit} 
\tag{indeterminate form}

You are given the following information: 
	\begin{enumerate}[label=\emph{(\roman*)}]
	\item		$\lim_{x\to 1} f(x) = 0$
	\item		$\lim_{x\to 1} g(x) = 0$
	\item		$\lim_{x\to 1} \frac{f(x)}{g(x)} = 2$
	\end{enumerate}
What can be said about the relative sizes of $f(x)$ and $g(x)$ as $x$ approaches $1$?	
\begin{prompt}
 \begin{multipleChoice}
  \choice{As $x$ is near $1$, both $f(x)$ and $g(x)$ are approximately $0$, but nothing else.}
  \choice[correct]{As $x$ is near $1$, both $f(x)$ and $g(x)$ are approximately $0$, but $f(x)$ is approximately $2\cdot g(x)$.}
  \choice{Nothing.}
  \choice{There is a removable discontinuity in $\frac{f(x)}{g(x)}$ at $x=1$.}
 \end{multipleChoice}
\end{prompt}
\end{exercise}
\end{document}