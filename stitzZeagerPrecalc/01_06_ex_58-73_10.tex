\documentclass{ximera}

\input{../preamble.tex}

\author{Carl Stitz \and Jeff Zeager \and Bart Snapp \and Matthew Carr}
\license{CC-By-SA-NC}
\acknowledgement{http://www.stitz-zeager.com/}

\begin{document}
\begin{exercise}


\outcome{Find domain and range.}
\outcome{Define and work with inverse functions.}
\outcome{Determine where a function is positive or negative}
\outcome{Find the intervals where a function is increasing or decreasing}

\tag{domain}
\tag{inverse}
\tag{increasing}
\tag{decreasing}

Now use the graph of $y = f(x)$ given below to answer the question.
\begin{image}
  \begin{tikzpicture}
    \begin{axis}[
        xmin=-5, xmax=5, ymin=-6,ymax=6,    
        unit vector ratio*=1 1 1,
        axis lines =middle, xlabel=$x$, ylabel=$y$,
        every axis y label/.style={at=(current axis.above origin),anchor=south},
        every axis x label/.style={at=(current axis.right of origin),anchor=west},
        xtick={-5,...,5}, ytick={-5,...,5},
        %xticklabels={-4,,-2,,0,,2,,4,,6}, yticklabels={,-2,,0,,2,,4,,6,,8,,10},
        grid=major,width=4in,
        grid style={dashed, gridColor},
      ]
      \addplot[color=penColor,very thick,smooth,domain=-4:4] {5*sin(x*180/4)};
      \addplot[color=penColor,fill=background,only marks,mark=*] coordinates{(2,5)};  %% open hole
      \addplot[color=penColor,fill=penColor,only marks,mark=*] coordinates{(2,3)};  %% closed hole
      
    \end{axis}
  \end{tikzpicture}
\end{image}


List the intervals where $f$ is increasing written from left to right. If none exist, write DNE. \[[\answer{-2},\answer{2})\]

\end{exercise}
\end{document}