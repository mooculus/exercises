\documentclass{ximera}
\input{../preamble.tex}
\author{Steven Gubkin}
\license{Creative Commons 3.0 By-NC}
\begin{document}
\begin{exercise}


\tag{integral}

\begin{warning}
	This question is challenging.
\end{warning}

Let $f$ be a function.  Consider the function $A(b) = \frac{1}{b} \int_0^b f(t)\d t $, which gives the average of $f$ on the interval $[0,b]$.  

\begin{exercise}
Show that if $A$ has a local maximum at $x=t$, then $A(t) = f(t)$.   Can you make sense of this result intuitively?
\end{exercise}

\begin{exercise}
	Show that if $f(x) = A(x)$ for all $x >0$, then $f$ must be a constant function.
\end{exercise}

\begin{exercise}
	Suppose $f(x)  = x(1-x)$.  For what value of $b$ is $A(b)$ maximized?

	\begin{prompt}
		$b = \frac{9-\sqrt{33}}{4}$
	\end{prompt}
\end{exercise}

\end{exercise}
\end{document}