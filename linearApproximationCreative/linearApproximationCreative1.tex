\documentclass{ximera}
\input{../preamble.tex}
\author{Steven Gubkin}
\license{Creative Commons 3.0 By-NC}
\begin{document}
\begin{exercise}

\outcome{Define linear approximation as an application of the tangent to a curve.}
 \outcome{Find the linear approximation to a function at a point and use it to approximate the function value.}
 \outcome{Identify when a linear approximation can be used.}

\tag{derivative}

Here is a cool idea for approximating $e^x$:

We know that $e^x = \left(e^\frac{x}{n}\right)^n$, by a familiar law of exponentials.  For a fixed value of $x$, if $n$ is large enough, then $\frac{x}{n}$ will be very small.

First use the linear approximation to $y=e^x$ at $x=0$ to give an approximation for $e^\frac{x}{n}$, and then use the idea above to leverage this into an approximtion of $e^x$.

The approximation you recieve may be familiar to you!

\begin{prompt}
$$e^\frac{x}{n} \approx \answer{1+\frac{x}{n}}$$

so

$$
e^x \approx \answer{(1+\frac{x}{n})^n}
$$
\end{prompt}

\end{exercise}
\end{document}