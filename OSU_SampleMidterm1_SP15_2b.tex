\documentclass{ximera}

\input{preamble.tex}

\author{Matthew Carr}
%% BADBAD
%% From https://math.osu.edu/sites/math.osu.edu/files/Math1151_SampleMidterm1_SP15.pdf
%% License?
%% Acknowledgement?
\license{BADBAD}
\acknowledgement{BADBAD}

\begin{document}
\begin{exercise}

\outcome{Calculate limits using the limit laws.}
\outcome{Calculate limits of the form 0/0.}

\tag{limits}

Find
\[
\lim_{x\to5}\left(\frac{\sqrt{x-4}-1}{x-5}\right)
\begin{prompt}
= \answer{6}
\end{prompt}
\]

\begin{hint}
Multiply the numerator and denominator by the conjugate of $\sqrt{x-4}-1$, namely, $\sqrt{x-4}+1$.
\end{hint}
\begin{hint}
$(\sqrt{x-4}-1)(\sqrt{x-4}+1)=x-4-1=x-5$. So $\frac{\sqrt{x-4}-1}{x-5}\cdot\frac{\sqrt{x-4}+1}{\sqrt{x-4}+1}=\frac{x-5}{(x-5)(\sqrt{x-4}+1)}=\frac{1}{\sqrt{x-4}+1}$.
\end{hint}
\begin{hint}
Then $\lim_{x\to5}\frac{1}{\sqrt{x-4}+1}=\frac{1}{2}$.
\end{hint}

\end{exercise}
\end{document}