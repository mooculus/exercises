\documentclass{ximera}
\input{preamble.tex}
\author{Bart Snapp}
\license{Creative Commons 3.0 By-NC}
\begin{document}
\begin{exercise}

\outcome{Understand the Squeeze Theorem and how it can be used to find limit values.}
\outcome{Calculate limits using the Squeeze Theorem.}
\tag{limit}
\tag{squeeze theorem}

Consider:
\[
\lim_{w\to 0} \left(-\sin \left(\frac{1}{w}\right) \sin (w)\right)
\]
A good way to compute this limit would be to use \wordChoice{\choice{limit laws}\choice{indeterminate forms}\choice[correct]{the Squeeze Theorem}\choice{the Intermediate Value Theorem}}.
\begin{exercise}
List two functions $g$ and $h$ such that
\[
g(w)\le -\sin \left(\frac{1}{w}\right) \sin (w) \le h(w)
\]
for all $w$ except for $w=\answer{0}$ on some interval containing $w=0$.
\[
g(w)=\answer{-\left| \sin (w)\right|}\qquad h(w) =\answer{\left| \sin (w)\right|}
\]
\begin{exercise}
Compute:
\[
\lim_{w \to 0}-\left| \sin (w)\right| = \answer{0}\qquad \lim_{w\to 0}\left| \sin (w)\right| = \answer{0}
\]
\begin{exercise}
By the Squeeze Theorem:
\[
\lim_{w\to 0} \left(-\sin \left(\frac{1}{w}\right) \sin (w)\right) = \answer{0}
\]
\end{exercise}
\end{exercise}
\end{exercise}
\end{exercise}
\end{document}