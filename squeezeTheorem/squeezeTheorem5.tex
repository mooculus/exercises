\documentclass{ximera}
\input{preamble.tex}
\author{Bart Snapp}
\license{Creative Commons 3.0 By-NC}
\begin{document}
\begin{exercise}

\outcome{Understand the Squeeze Theorem and how it can be used to find limit values.}
\outcome{Calculate limits using the Squeeze Theorem.}
\tag{limit}
\tag{squeeze theorem}

Consider:
\[
\lim_{u\to -2} \left(\sqrt[3]{u+2} \tan ^{-1}\left(\frac{1}{u+2}\right)\right)
\]
A good way to compute this limit would be to use \wordChoice{\choice{limit laws}\choice{indeterminate forms}\choice[correct]{the Squeeze Theorem}\choice{the Intermediate Value Theorem}}.
\begin{exercise}
List two functions $g$ and $h$ such that
\[
g(u)\le \sqrt[3]{u+2} \tan ^{-1}\left(\frac{1}{u+2}\right) \le h(u)
\]
for all $u$ except for $u=\answer{-2}$ on some interval containing $u=-2$.
\[
g(u)=\answer{-\frac{1}{2} \pi  \sqrt[3]{\left| u+2\right| }}\qquad h(u) =\answer{\frac{1}{2} \pi  \sqrt[3]{\left| u+2\right| }}
\]
\begin{exercise}
Compute:
\[
\lim_{u \to -2}-\frac{1}{2} \pi  \sqrt[3]{\left| u+2\right| } = \answer{0}\qquad \lim_{u\to -2}\frac{1}{2} \pi  \sqrt[3]{\left| u+2\right| } = \answer{0}
\]
\begin{exercise}
By the Squeeze Theorem:
\[
\lim_{u\to -2} \left(\sqrt[3]{u+2} \tan ^{-1}\left(\frac{1}{u+2}\right)\right) = \answer{0}
\]
\end{exercise}
\end{exercise}
\end{exercise}
\end{exercise}
\end{document}