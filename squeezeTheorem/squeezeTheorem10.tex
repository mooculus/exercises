\documentclass{ximera}
\input{preamble.tex}
\author{Steven Gubkin}
\license{Creative Commons 3.0 By-NC}
\begin{document}
\begin{exercise}

\outcome{Understand the Squeeze Theorem and how it can be used to find limit values.}
\outcome{Calculate limits using the Squeeze Theorem.}
\tag{limit}
\tag{squeeze theorem}

Let $f$ be a function which satisfies the following inequality:

\[
	A(x) \leq 3xf(x)+1 \leq B(x)
\]

for all real $x$.

Say we know that $\lim_{x \to 7} A(x) = \lim_{x \to 7} B(x)$.

\begin{multipleChoice}
	\choice{We can use this information to determine $\lim_{x \to 7}$ f(x)}
	\choice[correct]{We can conclude that $\lim_{x \to 7} f(x)$ exists, but cannot determine its value}
	\choice{We do not know whether $\lim_{x \to 3^+}$ exists}
\end{multipleChoice}
\end{exercise}
\end{document}