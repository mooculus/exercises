\documentclass{ximera}
\input{preamble.tex}
\author{Bart Snapp}
\license{Creative Commons 3.0 By-NC}
\begin{document}
\begin{exercise}

\outcome{Understand the Squeeze Theorem and how it can be used to find limit values.}
\outcome{Calculate limits using the Squeeze Theorem.}
\tag{limit}
\tag{squeeze theorem}

Consider:
\[
\lim_{v\to 0} \left(-v^2 \sin \left(\frac{3}{v}\right)\right)
\]
A good way to compute this limit would be to use \wordChoice{\choice{limit laws}\choice{indeterminate forms}\choice[correct]{the Squeeze Theorem}\choice{the Intermediate Value Theorem}}.
\begin{exercise}
List two functions $g$ and $h$ such that
\[
g(v)\le -v^2 \sin \left(\frac{3}{v}\right) \le h(v)
\]
for all $v$ on some interval containing $v=0$.
\[
g(v)=\answer{-\left| v\right| ^2}\qquad h(v) =\answer{\left| v\right| ^2}
\]
\begin{exercise}
Compute:
\[
\lim_{v \to 0}-\left| v\right| ^2 = \answer{0}\qquad \lim_{v\to 0}\left| v\right| ^2 = \answer{0}
\]
\begin{exercise}
By the Squeeze Theorem:
\[
\lim_{v\to 0} \left(-v^2 \sin \left(\frac{3}{v}\right)\right) = \answer{0}
\]
\end{exercise}
\end{exercise}
\end{exercise}
\end{exercise}
\end{document}