\documentclass{ximera}
\input{preamble.tex}
\author{Bart Snapp}
\license{Creative Commons 3.0 By-NC}
\begin{document}
\begin{exercise}

\outcome{Understand the Squeeze Theorem and how it can be used to find limit values.}
\outcome{Calculate limits using the Squeeze Theorem.}
\tag{limit}
\tag{squeeze theorem}

Consider:
\[
\lim_{x\to -1} \left((x+1)^3 \cos \left(\frac{1}{x+1}\right)\right)
\]
A good way to compute this limit would be to use \wordChoice{\choice{limit laws}\choice{indeterminate forms}\choice[correct]{the Squeeze Theorem}\choice{the Intermediate Value Theorem}}.
\begin{exercise}
List two functions $g$ and $h$ such that
\[
g(x)\le (x+1)^3 \cos \left(\frac{1}{x+1}\right) \le h(x)
\]
for all $x$ except for $x=\answer{-1}$ on some interval containing $x=-1$.
\[
g(x)=\answer{-\left| x+1\right| ^3}\qquad h(x) =\answer{\left| x+1\right| ^3}
\]
\begin{exercise}
Compute:
\[
\lim_{x \to -1}-\left| x+1\right| ^3 = \answer{0}\qquad \lim_{x\to -1}\left| x+1\right| ^3 = \answer{0}
\]
\begin{exercise}
By the Squeeze Theorem:
\[
\lim_{x\to -1} \left((x+1)^3 \cos \left(\frac{1}{x+1}\right)\right) = \answer{0}
\]
\end{exercise}
\end{exercise}
\end{exercise}
\end{exercise}
\end{document}