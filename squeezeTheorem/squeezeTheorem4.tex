\documentclass{ximera}
\input{preamble.tex}
\author{Bart Snapp}
\license{Creative Commons 3.0 By-NC}
\begin{document}
\begin{exercise}

\outcome{Understand the Squeeze Theorem and how it can be used to find limit values.}
\outcome{Calculate limits using the Squeeze Theorem.}
\tag{limit}
\tag{squeeze theorem}

Consider:
\[
\lim_{z\to 0} \left(z^3 e^{\tan ^{-1}\left(\frac{2}{z}\right)}\right)
\]
A good way to compute this limit would be to use \wordChoice{\choice{limit laws}\choice{indeterminate forms}\choice[correct]{the Squeeze Theorem}\choice{the Intermediate Value Theorem}}.
\begin{exercise}
List two functions $g$ and $h$ such that
\[
g(z)\le z^3 e^{\tan ^{-1}\left(\frac{2}{z}\right)} \le h(z)
\]
for all $z$ on some interval containing $z=0$.
\[
g(z)=\answer{e^{-\pi /2} \left| z\right| ^3}\qquad h(z) =\answer{e^{\pi /2} \left| z\right| ^3}
\]
\begin{exercise}
Compute:
\[
\lim_{z \to 0}e^{-\pi /2} \left| z\right| ^3 = \answer{0}\qquad \lim_{z\to 0}e^{\pi /2} \left| z\right| ^3 = \answer{0}
\]
\begin{exercise}
By the Squeeze Theorem:
\[
\lim_{z\to 0} \left(z^3 e^{\tan ^{-1}\left(\frac{2}{z}\right)}\right) = \answer{0}
\]
\end{exercise}
\end{exercise}
\end{exercise}
\end{exercise}
\end{document}