\documentclass{ximera}
\input{preamble.tex}
\author{Bart Snapp}
\license{Creative Commons 3.0 By-NC}
\begin{document}
\begin{exercise}

\outcome{Understand the Squeeze Theorem and how it can be used to find limit values.}
\outcome{Calculate limits using the Squeeze Theorem.}
\tag{limit}
\tag{squeeze theorem}

Consider:
\[
\lim_{x\to 0} \left(\sqrt[3]{x} e^{\sin \left(\frac{1}{x}\right)}\right)
\]
A good way to compute this limit would be to use \wordChoice{\choice{limit laws}\choice{indeterminate forms}\choice[correct]{the Squeeze Theorem}\choice{the Intermediate Value Theorem}}.
\begin{exercise}
List two functions $g$ and $h$ such that
\[
g(x)\le \sqrt[3]{x} e^{\sin \left(\frac{1}{x}\right)} \le h(x)
\]
for all $x$ on some interval containing $x=0$.
\[
g(x)=\answer{\frac{\sqrt[3]{\left| x\right| }}{e}}\qquad h(x) =\answer{e \sqrt[3]{\left| x\right| }}
\]
\begin{exercise}
Compute:
\[
\lim_{x \to 0}\frac{\sqrt[3]{\left| x\right| }}{e} = \answer{0}\qquad \lim_{x\to 0}e \sqrt[3]{\left| x\right| } = \answer{0}
\]
\begin{exercise}
By the Squeeze Theorem:
\[
\lim_{x\to 0} \left(\sqrt[3]{x} e^{\sin \left(\frac{1}{x}\right)}\right) = \answer{0}
\]
\end{exercise}
\end{exercise}
\end{exercise}
\end{exercise}
\end{document}