\documentclass{ximera}
\input{preamble.tex}
\author{Bart Snapp}
\license{Creative Commons 3.0 By-NC}
\begin{document}
\begin{exercise}

\outcome{Understand the Squeeze Theorem and how it can be used to find limit values.}
\outcome{Calculate limits using the Squeeze Theorem.}
\tag{limit}
\tag{squeeze theorem}

Consider:
\[
\lim_{z\to -1} \left(\sin (z+1) \left(-\tan ^{-1}\left(\frac{3}{z+1}\right)\right)\right)
\]
A good way to compute this limit would be to use \wordChoice{\choice{limit laws}\choice{indeterminate forms}\choice[correct]{the Squeeze Theorem}\choice{the Intermediate Value Theorem}}.
\begin{exercise}
List two functions $g$ and $h$ such that
\[
g(z)\le \sin (z+1) \left(-\tan ^{-1}\left(\frac{3}{z+1}\right)\right) \le h(z)
\]
for all $z$ except for $z=\answer{-1}$ on some interval containing $z=-1$.
\[
g(z)=\answer{-\frac{1}{2} \pi  \left| \sin (z+1)\right|}\qquad h(z) =\answer{\frac{1}{2} \pi  \left| \sin (z+1)\right|}
\]
\begin{exercise}
Compute:
\[
\lim_{z \to -1}-\frac{1}{2} \pi  \left| \sin (z+1)\right| = \answer{0}\qquad \lim_{z\to -1}\frac{1}{2} \pi  \left| \sin (z+1)\right| = \answer{0}
\]
\begin{exercise}
By the Squeeze Theorem:
\[
\lim_{z\to -1} \left(\sin (z+1) \left(-\tan ^{-1}\left(\frac{3}{z+1}\right)\right)\right) = \answer{0}
\]
\end{exercise}
\end{exercise}
\end{exercise}
\end{exercise}
\end{document}