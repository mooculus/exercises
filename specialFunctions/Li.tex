\documentclass{ximera}
\input{../preamble.tex}
\author{Steven Gubkin}
\license{Creative Commons 3.0 By-NC}
\begin{document}
\begin{exercise}

\tag{integral}

The \textbf{Offset Logarithmic Intergral} function is defined by

\[
\mathrm{Li}(x) = \int_2^x \frac{1)}{\ln(t)} \d t
\]

This function is very important in Analytic Number Theory.  It turns out that $Li(x)$ is a very good approximation for the number of prime numbers less than $x$. However, it cannot be expressed as an elemenary function (you cannot write it down in terms of rational functions, trig functions, exponentials, etc without an integral).

\begin{exercise}
	What is $\lim_{x \to \infty} Li(x)$?  Explain how you know.
	
	\begin{prompt}
	\[
	\lim_{x \to \infty} Li(x) = \answer{\infty}
	\]
	\end{prompt}
\end{exercise}


Here is a table of values for $Si(x)$:

\[
\begin{array}{c|c}
 x & \mathrm{Li}(x)\\ \hline
3 &  1.118\\
4 & 1.922\\
5 & 2.589\\
6 & 3.177\\
7& 3.712\\
8 & 4.209\\
9 & 4.676
\end{array}
\]


Using this table of values, evaluate the following definite integrals:

\begin{exercise}
	\[
	\int_2^4 \frac{1}{\ln(3+t)} \d t \begin{prompt} = \answer{1.123} \end{prompt} 
	\]
\end{exercise}

\begin{exercise}
	\[
	\int_2^3 \frac{t}{\ln(t)} \d t \begin{prompt} = \answer{2.754} \end{prompt} 
	\]
\end{exercise}

\end{exercise}
\end{document}