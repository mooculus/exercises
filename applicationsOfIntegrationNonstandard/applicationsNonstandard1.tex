\documentclass{ximera}
\input{../preamble.tex}
\author{Steven Gubkin}
\license{Creative Commons 3.0 By-NC}
\begin{document}

\begin{exercise}

\tag{integral}

A surface has is formed by revolving the curve $y = \frac{1}{x}$ from $x=1$ to $x=4$ about the $x$ -axis.  The $x$ axis is measuring distances in meters.  This surface has a variable density of $x^2 \unit{g}/\unit{m}^2$.  Set up and numerically evaluate an integral to find the mass of the surface in grams.  Report your answer to two decimal places.

\begin{hint}
	Accumulate the mass of infinitely many infinitesimal frustums.
\end{hint}

\begin{hint}
	The area of the frustum based at $x$ and with width $dx$ is $2\pi \frac{1}{x} \sqrt{1+\left(\frac{\d}{\d x} x^{-1}\right)^2} \d x = 2\pi \frac{1}{x} \sqrt{1+\frac{1}{x^2}} \d x$
\end{hint}

\begin{hint}
	Since the density of this frustum is $x^2$, the total mass of this frustum is $2 \pi x \sqrt{1+\frac{1}{x^2}} \d x$
\end{hint}

\begin{hint}
	We need to accumulate these infinitesimal masses from $x=1$ to $x=4$, so the total mass is
	
	\[
	\int_1^4 2 \pi x \sqrt{1+\frac{1}{x^2}} \d x \approx 51.18
	\]
\end{hint}

\begin{prompt}
	\[
	\textrm{Total Mass} = \answer{51.18} \unit{g}
	\]
\end{prompt}
\end{exercise}
\end{document}