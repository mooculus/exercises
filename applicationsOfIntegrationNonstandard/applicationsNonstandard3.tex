\documentclass{ximera}
\input{../preamble.tex}
\author{Steven Gubkin}
\license{Creative Commons 3.0 By-NC}
\begin{document}

\begin{exercise}

\tag{integral}

The Van der Waals force between two atoms is proportional to $r^6$, where $r$ is the distance between them.  This force, while small, is responsible for many interesting phenomena, including the ability of geckos to cling to glass.

Assume that two atoms obey $F(r) = \frac{Q}{r^6}$, where $r$ is the distance measured in nanometers ($\unit{nm}$).

How much work is done, in nanoNewtons ($\unit{nN}$), by moving the atoms from a distance of  $3 \unit{nm}$ to $5 \unit{nm}$?

\begin{hint}
The work done by moving the atoms $\d x \unit{nm}$ when they are $x \unit{nm}$ apart is $\d W = \frac{c}{x^6} \d x$.
\end{hint}

\begin{hint}
Accumulating these infinitesimal amounts of work from $x = 3$ to $x = 5$ yields

\begin{align*}
\textrm{Work} &=  \int_3^5 \frac{Q}{x^6} \d x\\
	&= \eval{\frac{-Q}{7x^7}}_3^5\\
	&=\frac{Q}{7(3^7)} - \frac{Q}{7(5)^5}\\
	&\approx -0.0005860609 Q\\
	&\approx 5.86 \times 10^{-4} Q
\end{align*}
\end{hint}

\begin{prompt}
	Express your answer is scientific notation, where the mantissa has two decimal places.
	
	\[
	\textrm{Work} = \answer{5.84} \times 10^{\answer{4}} Q \unit{nN}
	\]
\end{prompt}

\end{exercise}
\end{document}