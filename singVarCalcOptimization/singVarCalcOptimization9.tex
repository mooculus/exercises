\documentclass{ximera}
\input{../preamble.tex}
\author{Bart Snapp}
\license{Creative Commons 3.0 By-NC}
\acknowledgement{https://www.whitman.edu/mathematics/calculus/}
\begin{document}
\begin{exercise}
  \outcome{Interpret an optimization problem as the procedure used to make a system or design as effective or functional as possible.}
  \outcome{Set up an optimization problem by identifying the objective function and appropriate constraints.}
  \outcome{Solve optimization problems by finding the appropriate absolute extremum.}
  \outcome{Solve basic word problems involving maxima or minima.}

  \tag{extrema}
  \tag{constrained optimization}
  \tag{first derivative test}
  \tag{second derivative test}

  You want to make cylindrical containers to hold 1 liter using the
  least amount of construction material.  The side is made from a
  rectangular piece of material, and this can be done with no material
  wasted.  However, the top and bottom are cut from squares of side
  $2r$, so that $2(2r)^2=8r^2$ of material is needed (rather than
  $2\pi r^2$, which is the total area of the top and bottom).  Find
  the dimensions of the container using the least amount of material.
  \begin{prompt}
  \[
  \text{radius}=\answer{5}\qquad\text{height}=\answer{40/\pi}
  \]
  \end{prompt}
\end{exercise}
\end{document}
