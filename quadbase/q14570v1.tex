\documentclass{ximera}
%\input{../preamble.tex}
\author{Emma Smith Zbarsky}
\license{Creative Commons Attribution 3.0 Unported}
\acknowledgement{https://quadbase.org/questions/q14570v1}
\begin{document}

\begin{exercise}

Consider the function $y=-x^2+10$, shown below with several tangent
lines: \{img:newtonsmethod-ost-opt2\_1.png\} If these tangent lines were
created using Newton's Method, what order were they created in?


\begin{hint}
The key here is to think about what Newton's method is doing. Newton's
method is iteratively searching for roots, zeros, of a function and each
step should bring you closer to a root.
\end{hint}


\begin{hint}
\begin{image}\includegraphics{newtonsmethod-ost-opt2.png}\end{image}



Here, we start with an initial guess of a 0.5. That gives us a
subsequent guess of just over 10 (the blue line) followed by about 5.7
(the green line) and lastly by about 3.7 (the black line). Since it is
converging to the root $\sqrt{10} \simeq 3.1623$, we haven't done too
badly.
\end{hint}


\begin{multipleChoice}
\choice{First green, then black, then blue.}
\choice[correct]{First blue, then green, then black.}
\choice{First blue, then black, then green.}
\choice{First black, then green, then blue.}
\choice{First green, then blue, then black.}
\end{multipleChoice}

\end{exercise}
\end{document}
