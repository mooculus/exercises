\documentclass{ximera}
%\input{../preamble.tex}
\author{Emma Smith Zbarsky}
\license{Creative Commons Attribution 3.0 Unported}
\acknowledgement{https://quadbase.org/questions/q14253v1}
\begin{document}

\begin{exercise}

Compute the derivative with respect to $T$ of the ideal gas law:
\[V(T) = \frac{nRT}{P},\] where $V$ is the volume, $n$ is a constant
giving the number of moles of the gas, $R$ is the universal gas
constant, and $P$ is a constant pressure.

Note that $\frac{dV}{dT}$ expresses how the volume of the gas changes as
the temperature changes. You should particularly pay attention to the
sign of your result--does gas expand or contract as it is warmed up?


\begin{hint}
This is a simple power rule derivative. The complications all arise from
the use of actual named constants, $n$, $R$, and $P$.
\end{hint}


\begin{hint}
Because $n$, $R$, and $P$ are all constant terms, we could write this
problem as
\[\frac{dV}{dT} = \frac{d}{dT}\left(\left(\frac{nR}{P}\right)T\right).\]
Using the first linearity rule for constants, this becomes
\begin{align*}
\frac{dV}{dT} &= \left(\frac{nR}{P}\right)\frac{d}{dT}\left(T\right) \\
&= \left(\frac{nR}{P}\right)(1) \\
&= \boxed{\frac{nR}{P}}
\end{align*}
\end{hint}


\begin{multipleChoice}
\choice{$V'(T) = \frac{nT}{P}$}
\choice{$V'(T) = 0$}
\choice{$V'(T) = \frac{RT}{P}$}
\choice[correct]{$V'(T) = \frac{nR}{P}$}
\choice{$V'(T) = -\frac{nRT}{P^2}$}
\end{multipleChoice}

\end{exercise}
\end{document}
