\documentclass{ximera}
%\input{../preamble.tex}
\author{Emma Smith Zbarsky}
\license{Creative Commons Attribution 3.0 Unported}
\acknowledgement{https://quadbase.org/questions/q14279v1}
\begin{document}

\begin{exercise}

Calculate the derivative with respect to $\theta$ of
\[r(\theta) = \left(8\pi\theta-\frac{3\pi}{2}\right)\tan(\theta+\pi/2).\]


\begin{hint}
This is a mixed derivative rule problem. There is a product rule and a
chain rule.
\end{hint}


\begin{hint}
Letting $r(\theta) = ab$ we could take $a = 8\pi\theta-\frac{3\pi}{2}$
and $b = \tan(\theta+\pi/2)$. Then $a'=8\pi$ is a simple power rule
problem but $b = f(g(\theta))$ with $f = \tan(g)$ and $g = \theta+\pi/2$
involves a chain rule to calculate
\[b' = \sec^2(\theta+\pi/2)\cdot(1) = \sec^2(\theta+\pi/2).\] Putting it
all together, we have: \begin{align*}
r'(\theta) &= 8\pi\tan(\theta+\pi/2) + \left(8\pi\theta-\frac{3\pi}{2}\right)\sec^2(\theta+\pi/2)
\end{align*}
\end{hint}


\begin{multipleChoice}
\choice{$r'(\theta) = 8\pi\sec^2(1)$}
\choice{$r'(\theta) = 8\pi\tan(\theta+\pi/2)-\left(8\pi\theta-\frac{3\pi}{2}\right)\sec^2(\theta+\pi/2)$}
\choice{$r'(\theta) = 8\pi\sec^2(\theta+\pi/2)$}
\choice[correct]{$r'(\theta) = 8\pi\tan(\theta+\pi/2)+\left(8\pi\theta-\frac{3\pi}{2}\right)\sec^2(\theta+\pi/2)$}
\choice{$r'(\theta) = 0$}
\end{multipleChoice}

\end{exercise}
\end{document}
