\documentclass{ximera}
%\input{../preamble.tex}
\author{Emma Smith Zbarsky}
\license{Creative Commons Attribution 3.0 Unported}
\acknowledgement{https://quadbase.org/questions/q13591v1}
\begin{document}

\begin{exercise}

A graph of $p(t)$ must be increasing, decreasing, constant, or periodic?


\begin{hint}
This is a real world problem describing a function. You are expected to
use your knowledge of entropy and the second law of thermodynamics to
solve it.
\end{hint}


\begin{hint}
A graph of $p(t)$ looks like:



\begin{image}\includegraphics{salmonpop.jpg}\end{image}



Effectively, once a salmon has died it cannot come back to life so the
function is increasing. If we were to look at the number of living
salmon in the Columbia River basin over long time scale (say 10-20
years) we would expect a periodic function as the fish hatch, leave the
basin, return, spawn and die.
\end{hint}


\begin{multipleChoice}
\choice{Constant}
\choice{Decreasing}
\choice[correct]{Increasing}
\choice{Periodic}
\end{multipleChoice}

\end{exercise}
\end{document}
