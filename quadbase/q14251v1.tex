\documentclass{ximera}
%\input{../preamble.tex}
\author{Emma Smith Zbarsky}
\license{Creative Commons Attribution 3.0 Unported}
\acknowledgement{https://quadbase.org/questions/q14251v1}
\begin{document}

\begin{exercise}

Calculate the derivative with respect to $r$ of
\[G(r) = \frac{g m_1m_2}{r^2}\] where $g$, $m_1$ and $m_2$ are
constants.

Note: This is the force due to gravity on two objects of masses $m_1$
and $m_2$ that are a distance $r$ apart. $g$ is the gravitational
constant.


\begin{hint}
This is a power rule problem, though it could also be done with the
quotient rule. You will need to use either
$\frac{d}{dr}\left(r^n\right) = nr^{n-1}$ or
\[\frac{d}{dr}\left(\frac{p(r)}{q(r)}\right) = \frac{p'(r)q(r)-p(r)q'(r)}{\left(q(r)\right)^2}\]
to solve it.
\end{hint}


\begin{hint}
Using the power rule, we have \begin{align*}
G(r) &= \frac{gm_1m_2}{r^2} = gm_1m_2r^{-2} \\
G'(r) &= gm_1m_2(-2)r^{-2-1} = -2gm_1m_2r^{-3} \\
&= \boxed{\frac{-2gm_1m_2}{r^3}}\end{align*}

Using the quotient rule
\[\frac{d}{dr}\left(\frac{p(r)}{q(r)}\right) = \frac{p'(r)q(r)-p(r)q'(r)}{\left(q(r)\right)^2}\]
with $p(r) = gm_1m_2$ and $q(r) = r^2$ we have \begin{align*}
p'(r) &= 0 \\
q'(r) &= 2r^1 = 2r\\
G'(r) &= \frac{0(r^2)-gm_1m_2(2r)}{(r^2)^2} \\
&= -\frac{2gm_1m_2r}{r^4} \\
&= \boxed{-\frac{2gm_1m_2}{r^3}}
\end{align*}
\end{hint}


\begin{multipleChoice}
\choice[correct]{$G'(r) = -\frac{2gm_1m_2}{r^3}$}
\choice{$G'(r) = 0$}
\choice{$G'(r) = \frac{2gm_1m_2}{r^3}.$}
\choice{$G'(r) = -\frac{gm_1m_2}{r^3}$}
\choice{$G'(r) = -\frac{gm_1m_2}{r^4}$}
\end{multipleChoice}

\end{exercise}
\end{document}
