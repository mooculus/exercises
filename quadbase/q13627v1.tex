\documentclass{ximera}
%\input{../preamble.tex}
\author{Emma Smith Zbarsky}
\license{Creative Commons Attribution 3.0 Unported}
\acknowledgement{https://quadbase.org/questions/q13627v1}
\begin{document}

\begin{exercise}

If $\displaystyle \lim_{x\to 5} f(x) = 10$, then the function $f(x)$ is
continuous at $x=5$ with $f(5)=10$.


\begin{hint}
A limit exists if the right and left one-sided limits exist and agree.
That does not imply that the function is continuous at that point.
\end{hint}


\begin{hint}
For example, consider the function \[f(x) = \frac{x^2-25}{x-5}.\] This
function does not exist at $x=5$ where it would take the value
$\displaystyle \frac{0}{0}.$ However, since it has a removable
discontinuity at $x=5$, we can see that
\[\lim_{x\to 5} f(x) = \lim_{x\to 5}\frac{(x+5)(x-5)}{x-5} = \lim_{x\to 5}\frac{x+5}{1} = 10.\]
\end{hint}


\begin{multipleChoice}
\choice{True}
\choice[correct]{False}
\end{multipleChoice}

\end{exercise}
\end{document}
