\documentclass{ximera}
%\input{../preamble.tex}
\author{Emma Smith Zbarsky}
\license{Creative Commons Attribution 3.0 Unported}
\acknowledgement{https://quadbase.org/questions/q14592v1}
\begin{document}

\begin{exercise}

True or False? In the limit, we have:
\[\lim_{n\to \infty} \sum_{i=1}^n f(x_i) \; \Delta x_i = \int_{x_0}^{x_n} f(x) \; dx\]
where $\Delta x_i = x_i-x_{i-1} = \frac{x_n-x_0}{n}.$


\begin{hint}
This is a question about the relationship between (right-endpoint)
Riemann sums and definite integrals.
\end{hint}


\begin{hint}
Since the equality of Riemann sums in the limit as the number of
rectangles goes to infinity can be considered the definition of a
definite integral over the same interval, this statement is true.
Another way to see this is to consider the relationship between
antiderivatives and definite integrals in the Fundamental Theorem of
Calculus. In that case it was clear that what a definite integral is
computing is the area under a curve, the same as that computed by a
limit of Riemann sums.
\end{hint}


\begin{multipleChoice}
\choice[correct]{True}
\choice{False}
\choice{Not enough data to tell}
\end{multipleChoice}

\end{exercise}
\end{document}
