\documentclass{ximera}
%\input{../preamble.tex}
\author{Emma Smith Zbarsky}
\license{Creative Commons Attribution 3.0 Unported}
\acknowledgement{https://quadbase.org/questions/q13626v1}
\begin{document}

\begin{exercise}

If $\displaystyle \lim_{x\to 5} f(x) = \infty$, then the limit exists
with value $\infty$.


\begin{hint}
This question is exploring the idea of what it means for a limit to
exist. Limits only exist when there is a single finite value approached
by the function at that point.
\end{hint}


\begin{hint}
In this case, the function $f(x)$ is unbounded above, or in other words
diverging to positive infinity, from both directions as $x \to 5$. We
write that the limit `equals' infinity, but that is a statement about
the direction of the divergence is occurring, not the existence of a
limit value. It means that no matter how large a value of $M$ we choose,
there is some small $\epsilon$ such that $f(x)>M$ for
$5-\epsilon \leq x \leq 5+\epsilon$.
\end{hint}


\begin{multipleChoice}
\choice[correct]{False}
\choice{True}
\end{multipleChoice}

\end{exercise}
\end{document}
