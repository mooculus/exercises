\documentclass{ximera}
%\input{../preamble.tex}
\author{Emma Smith Zbarsky}
\license{Creative Commons Attribution 3.0 Unported}
\acknowledgement{https://quadbase.org/questions/q14531v1}
\begin{document}

\begin{exercise}

Find and identify the extrema of $f(x) = \ln(x)-x.$


\begin{hint}
This is an extreme value problem. Because $f$ is continuous on its
domain, $x>0$, we can solve this by only checking the critical points of
$f$ using the second derivative test.
\end{hint}


\begin{hint}
\begin{align*}
f' &= \frac{1}{x}-1 \\
0 &= \frac{1}{x^*}-1 \\
& \Rightarrow x^* = 1 \\
&\\
f'' &= -\frac{1}{x^2} \\
f''(1) &= -\frac{1}{1^2} = -1 <0 \Rightarrow \mbox{ maximum}
\end{align*} There is one extreme value for $f(x) = \ln(x)-x$ -- a
(global) maximum at $x=1$. 

\begin{image}\includegraphics{extremalnx-x.jpg}\end{image}
\end{hint}


\begin{multipleChoice}
\choice[correct]{There is one extremum: a maximum at $x=1$.}
\choice{There are three extrema: a minimum at $x=0$, a maximum at $x=1$ and a
minimum at $x=\infty$.}
\choice{There are no extrema on $f$.}
\choice{There are two extrema: a minimum at $x=0$ and a maximum at $x=1$.}
\choice{There is one extremum: a minimum at $x=1$.}
\end{multipleChoice}

\end{exercise}
\end{document}
