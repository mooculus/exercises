\documentclass{ximera}
%\input{../preamble.tex}
\author{Emma Smith Zbarsky}
\license{Creative Commons Attribution 3.0 Unported}
\acknowledgement{https://quadbase.org/questions/q13648v1}
\begin{document}

\begin{exercise}

Say we have a hurricane with a circular shape. If the radius of the
storm is growing at 5 miles per hour, how fast is the area of the storm
growing when the storm is 120 miles in diameter?


\begin{hint}
Identify the relationship between the known and unknown variables and
differentiate it to relate the known and unknown rates.
\end{hint}


\begin{hint}
Name the variables:

*area = $A$ in square miles

*radius = $r$ in miles

*time = $t$ in hours

Known values: \[\frac{dr}{dt} = 5 \mbox{ miles per hour }\]
\[\mbox{diameter} = 120 \mbox{ miles} \Rightarrow r = 60 \mbox{ miles}\]

Known formulas (relate the variables):

\begin{itemize}
\itemsep1pt\parskip0pt\parsep0pt
\item
  Area of a circle: \[A = \pi r^2\]
\end{itemize}

Related rates (take the derivative with respect to time):
\[\frac{dA}{dt} = 2\pi r \cdot \frac{dr}{dt}\]

Plug in the known values:
\[\frac{dA}{dt} = 2\cdot \pi \cdot 60 \cdot 5 = 600 \pi \mbox{ square miles per hour}\]
\end{hint}


\begin{multipleChoice}
\choice{$300 \pi$ miles squared per hour}
\choice{$240 \pi$ miles squared per hour}
\choice{$10 \pi$ miles squared per hour}
\choice{$5\pi$ miles per hour}
\choice{$120\pi$ miles squared per hour}
\choice{$10 \pi$ miles per hour}
\choice[correct]{$600 \pi$ miles squared per hour}
\choice{$5\pi$ miles squared per hour}
\choice{$1200 \pi$ miles squared per hour}
\end{multipleChoice}

\end{exercise}
\end{document}
