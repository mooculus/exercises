\documentclass{ximera}
%\input{../preamble.tex}
\author{Emma Smith Zbarsky}
\license{Creative Commons Attribution 3.0 Unported}
\acknowledgement{https://quadbase.org/questions/q14535v1}
\begin{document}

\begin{exercise}

Consider a trip along the Mass Pike from the intersection with I-95 to
the Berkshires. You pass a toll booth on entering at the I-95
interchange at 12:35 pm. You pass a toll both when exiting to the
Berkshires at 2:05 pm. Having gone 116 miles, if the maximum speed limit
on the route is 70 miles per hour, do you deserve a ticket? Why or why
not?


\begin{hint}
The mean value theorem says that if $f(x)$ is defined and continuous on
the interval $[a,b]$, then there is at least one point $c \in (a,b)$
with $f'(c) = \frac{f(b)-f(a)}{b-a}.$
\end{hint}


\begin{hint}
Applying the mean value theorem in this case, we see that we traveled
\[\frac{116 \mbox{ miles}}{2:05-12:35 \mbox{ hours}} = \frac{116}{1.5} \mbox{ mph} = 77.\overline{3} \mbox{ mph}.\]
Therefore, if we consider our position as $f(t)$, there is some time $t$
between 12:35 pm and 2:05 pm when $f'(t) = 77.\overline{3}$ mph so we
must have been driving over the speed limit.
\end{hint}


\begin{multipleChoice}
\choice{No, you traveled at an average speed of just over 66 mph, so while you
may have exceeded the speed limit, the toll booth operators cannot prove
it.}
\choice[correct]{Yes, you traveled at an average speed of just over 77 mph, so you must
have exceeded the speed limit for a large part of your journey (or gone
extremely fast for a short time).}
\end{multipleChoice}

\end{exercise}
\end{document}
