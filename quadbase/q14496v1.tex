\documentclass{ximera}
%\input{../preamble.tex}
\author{Emma Smith Zbarsky}
\license{Creative Commons Attribution 3.0 Unported}
\acknowledgement{https://quadbase.org/questions/q14496v1}
\begin{document}

\begin{exercise}

Consider the ideal gas law, \[PV = nRT.\] How fast is the temperature of
a balloon changing when the temperature of the gas inside is 326.15
degrees K, the pressure is 101325 N/m$^2$, $R=8.31432$ Nm/(mol K),
$V=.25$ m$^3$ and $n=345$ mols if you are inflating the balloon at 1
m$^3$/min while the pressure increases at a rate of $1.135$ N/m$^2$ per
minute?

Give your answer to the nearest tenth of a degree per minute.


\begin{hint}
This is a related rates problem with a given equation. Take the
derivative with respect to time and plug in the given values to compute
$\frac{dT}{dt}$.
\end{hint}


\begin{hint}
\begin{align*}
\frac{dP}{dt}V + P\frac{dV}{dt} &= nR\frac{dT}{dt} \\
(1.135)(.25) + 101325(1) &= 345(8.31432) \frac{dT}{dt} \\
\frac{dT}{dt} &\simeq 35.3
\end{align*} So the temperature is increasing at a rate of $35.3$
degrees Kelvin per minute.
\end{hint}


\begin{multipleChoice}
\choice{$35.5$ degrees K per minute}
\choice{$35.2$ degrees K per minute}
\choice{$35.4$ degrees K per minute}
\choice{$35.6$ degrees K per minute}
\choice[correct]{$35.3$ degrees K per minute}
\end{multipleChoice}

\end{exercise}
\end{document}
