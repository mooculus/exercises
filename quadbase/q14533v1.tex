\documentclass{ximera}
%\input{../preamble.tex}
\author{Emma Smith Zbarsky}
\license{Creative Commons Attribution 3.0 Unported}
\acknowledgement{https://quadbase.org/questions/q14533v1}
\begin{document}

\begin{exercise}

Find, and identify, the extrema of the unit step function
\[H(t) = \begin{cases} 1 & t > 0 \\
\frac{1}{2} & t = 0 \\ 0 & t < 0 \end{cases}.\]

`'Note: $H(t)$ is also called the Heaviside unit step function. It
arises frequently in solving differential equations.''


\begin{hint}
This is an extreme value problem where every point is a critical point,
so think carefully about what it means to be a maximum or a minimum.
\end{hint}


\begin{hint}
Because \[H'(t) = \begin{cases} 0 & t \neq 0 \\
DNE & t = 0 \end{cases},\] every point is a critical point for $H$. The
second derivative is the same, however, so it is not a helpful test.
Using our definition of maximum, we see that $H(t^+)=1\geq H(t)$ for any
$t^+ > 0$ and any real number $t$, so each $t^+$ is a maximum.
Similarly, $H(t^-) = 0 \leq H(t)$ for any $t^- < 0$ and any real number
$t$ so each $t^-$ is a minimum. This only misses $H(0) = \frac{1}{2}$,
which is larger than 0 but smaller than 1.
\end{hint}


\begin{multipleChoice}
\choice[correct]{Every point $t < 0$ is a (global) minimum and every point $t > 0$ is a
(global) maximum.}
\choice{Because every point is a critical point, every value $t$ is an extremum
for $H(t)$.}
\choice{There are no extrema for $H(t)$.}
\end{multipleChoice}

\end{exercise}
\end{document}
