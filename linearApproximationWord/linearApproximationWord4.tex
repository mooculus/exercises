\documentclass{ximera}
\input{../preamble.tex}
\author{Steven Gubkin}
\license{Creative Commons 3.0 By-NC}
\begin{document}
\begin{exercise}

\outcome{Define linear approximation as an application of the tangent to a curve.}
 \outcome{Find the linear approximation to a function at a point and use it to approximate the function value.}
 \outcome{Identify when a linear approximation can be used.}

\tag{derivative}

You are studying a solar energy collection unit.  Your idea is to use a lens to focus light on a solar panel, to get greater efficiency.  In your theoretical model, the efficiency of the system is $E(r) = \frac{40e^{\frac{2}{r}}}{e^{\frac{2}{r}}+27}$, where $r$ is the radius of the focused light beam in centimeters.  Using a linear approximation, what gain in efficiency should you expect from reducing the radius from $1$ to $0.9$ centimeters? 

\begin{prompt}
	Using this model, we expect a gain of $\answer{1.34959} $ percent efficiency.
\end{prompt}

\end{exercise}
\end{document}