\documentclass{ximera}
\input{../preamble.tex}
\author{Steven Gubkin}
\license{Creative Commons 3.0 By-NC}
\begin{document}
\begin{exercise}

\outcome{Identify cross sections.}
\outcome{Compute volumes of shapes with arbitrary cross sections}
\outcome{Use the disk method to compute volumes.}
%\outcome{Use the washer method to compute volumes.}
%\outcome{Compute volumes of revolution around the $x$-axis.}
\outcome{Compute volumes of revolution around the $y$-axis.}
%\outcome{Compute volumes of revolution around arbitrary lines.}

\tag{integral}

Consider the region bounded by $y =\ln(x)$ , the $x$ axis, the $y$ axis, and the horizontal line $y=1$.  What is the volume of the solid obtained by revolving this region about the $y$ axis?


\begin{hint}
	Draw a picture!
\end{hint}

\begin{hint}
	Solving for $x$, we have $x = e^y$.  
\end{hint}

\begin{hint}
	We can decompose the solid into infinitesmal disks with width $dy$ and radius $e^y$.  The volume of each washer is $\pi (e^y)^2\d y$.  Summing these volumes from $y=0$ to $y=1$, we obtain

	\[
	\textrm{Volume} = \int_0^1 \pi (e^y)^2\d y
	\]
\end{hint}

\begin{hint}
	\begin{align*}
		 \int_0^1 \pi (e^y)^2\d y &= \pi \int_0^1e^{2y}\d y \\
			&=\frac{\pi}{2}\eval{e^{2y}}_0^1\\
			&=\frac{pi}{2}(e^2-e)
	\end{align*}
\end{hint}

\begin{prompt}
	\[
		\textrm{Volume} = \answer{\frac{pi}{2}(e^2-e)}
	\]
\end{prompt}

\end{exercise}
\end{document}