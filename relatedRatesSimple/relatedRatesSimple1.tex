\documentclass{ximera}
\input{../preamble.tex}
\author{Steven Gubkin}
\license{Creative Commons 3.0 By-NC}
\begin{document}

\begin{exercise}

\outcome{Solve basic related rates word problems.}
\outcome{Understand the process of solving related rates problems.}
\outcome{Calculate derivatives of expressions with multiple variables implicitly.}

\tag{derivative}

A sphere is expanding with time at a constant rate of $3 \frac{\textrm{m}^3}{\textrm{s}}$.  At a certain time, the radius  is $5 \textrm{m}$.  How fast is the surface area growing at that time?

\begin{hint}
	The volume of a sphere is $V = \frac{4}{3} \pi r^3$, and its surface area is $S = 4 \pi r^2$.
\end{hint}

\begin{hint}
	Differentiating  both equations implicitly with respect to $t$, we obtain

\[
\begin{cases}
\frac{\d V}{\d t} = 4 \pi r^2 \frac{\d r}{\d t} \\
\frac{\d S}{\d t} = 8\pi r \frac{\d r}{\d t}
\end{cases}
\]
\end{hint}

\begin{hint}
	Substituing what we know about these quantities, we have

\[
\begin{cases}
3= 4 \pi (5)^2 \frac{\d r}{\d t} \\
\frac{\d S}{\d t} = 8\pi (5) \frac{\d r}{\d t}
\end{cases}
\]
\end{hint}

\begin{hint}
	The first equation allows us to see that $\frac{\d r}{\d t}  = \frac{3}{100 \pi}$, which then implies, through the second equation, that $\frac{\d S}{\d t} = 1.2$
\end{hint}

\begin{prompt}
	$\frac{\d S}{\d t} = \answer{1.2}$
\end{prompt}

\end{exercise}

\end{document}
