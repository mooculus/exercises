\documentclass{ximera}
\input{../preamble.tex}
\author{Steven Gubkin}
\license{Creative Commons 3.0 By-NC}
\begin{document}

\begin{exercise}

\outcome{Solve basic related rates word problems.}
\outcome{Understand the process of solving related rates problems.}
\outcome{Calculate derivatives of expressions with multiple variables implicitly.}

\tag{derivative}

A right circular cone is growing.  As it grows, its height is remains equal to its diameter. The diameter is growing at a constant rate of $2 \frac{\textrm{units}}{\textrm{s}}$. At what rate is its volume growing at the time the diameter is $6 \textrm{units}$?


\begin{hint}
	If we let $V$be the volume, $r$ the radius, and $D$ the diameter,  and $h$ the height, then we know

\[
\begin{cases}
	D = 2r\\
	V = \frac{1}{3} \pi r^2 h\\
	h = D\\
\end{cases}
\]
\end{hint}

\begin{hint}
	While we could blindly differentiate, and everything would work out fine, it pays to do a little algebra first since we only care about the relationship between the volume and the diameter.  A little rearranging yields

\[
V = \frac{1}{3} \pi \left(\frac{D}{2}\right)^2D
\]

so

\[
V = \frac{1}{12} \pi D^3
\]
\end{hint}

\begin{hint}
	Differentiating with respect to time, we obtain

\[\frac{\d V}{\d t} = \frac{1}{4} \pi D^2 \frac{\d D}{\d t}\]
\end{hint}

\begin{hint}
	At the time of interest, we have

\[\frac{\d V}{\d t} = \frac{1}{4} \pi (6) (2) = 3\pi\]
\end{hint}

\begin{prompt}
	\[\frac{\d V}{\d t} = \answer{ 3\pi}\]
\end{prompt}

\end{exercise}

\end{document}
