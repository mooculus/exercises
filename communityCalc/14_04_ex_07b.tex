\documentclass{ximera}

\input{../preamble.tex}

\author{David Guichard \and Neal Koblitz \and H. Jerome Keisler \and Albert Scheller \and Barry Balof \and Mike Wills \and Matthew Carr}
\license{CC-By-SA-NC}

\acknowledgement{https://www.whitman.edu/mathematics/multivariable/}

\begin{document}
\begin{exercise}


%% BADBAD no outcomes

\tag{chain rule}

Consider the ideal gas law, given by $PV=nRT$, relating pressure, $P$, volume, $V$, and the temperature $T$ of $n$ moles of gas, where $R$ is the ideal gas constant. We can view $P$, $V$ and $T$ each as functions of the other two variables. 

If the volume of a gas is decreasing at a rate of $0.3\unit{L/min}$ and temperature is increasing at a rate of $0.5\unit{K/min}$, how fast is the pressure changing? Express your answer in terms of $P$, $V$, $n$ and $R$ alone.

\begin{prompt}
\[
\frac{d P}{d t}=\answer{\frac{n R+0.6 P}{2V}}
\]
\end{prompt}

\end{exercise}
\end{document}