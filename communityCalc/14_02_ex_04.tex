\documentclass{ximera}

\input{../preamble.tex}

\author{David Guichard \and Neal Koblitz \and H. Jerome Keisler \and Albert Scheller \and Barry Balof \and Mike Wills \and Matthew Carr}
\license{CC-By-SA-NC}

\acknowledgement{https://www.whitman.edu/mathematics/multivariable/}

\begin{document}
\begin{exercise}


%% BADBAD no outcomes

\tag{limits}

Find the following the limit. If it does not exist, write DNE. 

\[
\lim_{(x,y)\to(0,0)}\frac{x^4-y^4}{x^2+y^2}\begin{prompt}=\answer{0}\end{prompt}
\]

If you didn't guess the above limit, then you probably noticed that the numerator of the function enjoys a particularly nice property vis-\`a-vis the denominator. 

Write $x^4-y^4$ as a product of two functions where one of these functions satisfies a convenient relation with respect to the denominator, $x^2+y^2$. (In each pair of your parentheses, write terms with $x$'s first and terms with $y$'s next.)

\begin{prompt}
\[
x^4-y^4=(\answer{x^2}+\answer{y^2})(\answer{x^2}-\answer{y^2})
\]
\end{prompt}

\end{exercise}
\end{document}