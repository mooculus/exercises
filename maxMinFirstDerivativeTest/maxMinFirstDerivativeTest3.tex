\documentclass{ximera}
\input{../preamble.tex}
\author{Steven Gubkin}
\license{Creative Commons 3.0 By-NC}
\begin{document}

\begin{exercise}
\outcome{ Define a critical point.}
\outcome{ Find critical points.}
\outcome{ Define absolute maximum and absolute minimum.}
\outcome{ Find the absolute max or min of a continuous function on a closed interval.}
\outcome{ Define local maximum and local minimum.}
\outcome{ Compare and contrast local and absolute maxima and minima.}
\outcome{ Identify situations in which an absolute maximum or minimum is guaranteed.}
\outcome{ Classify critical points.}
\outcome{ State the First Derivative Test.}
\outcome{ Apply the First Derivative Test.}
\tag{derivative}

The function $f(x) =\frac{x^2-2x+1}{x(1+x^2)}$ has two critical points.    If we call these critical point $a$ and $b$, and order them such that $a < b$, then

$$
a = \answer{0}
$$

$$
b=\answer{1}
$$



On $(-\infty,a)$, $f$ is \wordChoice{\choice{increasing} \choice[correct]{decreasing}}

On $(a,b)$, $f$ is \wordChoice{\choice{increasing} \choice[correct]{decreasing}}

On $(b,\infty)$, $f$ is \wordChoice{\choice[correct]{increasing} \choice{decreasing}}


$x=a$ is a \wordChoice{\choice{Local max} \choice{Local Min} \choice[correct]{Neither}}

$x=b$ is a \wordChoice{\choice{Local max} \choice[correct]{Local Min} \choice{Neither}}

\end{exercise}
\end{document}
